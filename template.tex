\documentclass[10pt,aspectratio=169]{beamer}
\usepackage{poly}
\usepackage[useregional]{datetime2}
\usepackage{graphicx}  % for images
\usepackage{xcolor}    % for color
\usepackage{minted}    % for better code highlighting
\usepackage[T1]{fontenc}  % for better font encoding
\usepackage{inconsolata}  % fallback monospace font

% Minted configuration
\setminted{
    frame=none,
    fontsize=\small,
    linenos=true,
    numbersep=12pt,
    bgcolor=white,
    style=friendly,
    autogobble,
    breaklines,
    baselinestretch=1.2
}

% Add slide numbers to the footer
\setbeamertemplate{footline}[frame number]
\setbeamertemplate{navigation symbols}{}  % Remove navigation symbols

% Define a centered minted environment
\newenvironment{centeredcode}{\begin{center}}{\end{center}}

\title{Sistem Operasi (Judul mata Kuliah)}
\subtitle{Judul Pertemuan}
\author{Martin Clinton Manullang}
\institute{Program Studi Teknik Informatika}
\date{\today}

\begin{document}
\maketitle

\section{Sample Section}
% --- Frame: Formatting Examples ---
\begin{frame}{Formatting Examples}
    \framesubtitle{Subtitle Example}
    % Sample text with various formatting
    This is an example of \textbf{bold text}, \textit{italic text}, and \texttt{monospaced text}.
    
    \bigskip
    % Normal block
    \begin{block}{Block Title}
      This is a normal block with some dummy text.
    \end{block}
    
    % Alert block
    \begin{alertblock}{Alert Block}
      This is an \textbf{alert} block showing important information.
    \end{alertblock}
\end{frame}

% --- Frame: Two Column Example ---
\begin{frame}{Two Column Layout Example}
    \begin{columns}[T]
        \column{0.48\textwidth}
            \textbf{Left Column: Lists and Text}
            \begin{itemize}
                \item First item with \textit{italics}.
                \item Second item with \textbf{bold}.
                \item Third item with \texttt{typewriter font}.
            \end{itemize}
        \column{0.48\textwidth}
            \textbf{Right Column: Table and Graphic}
            \vspace{0.5em}
            % Sample table
            \begin{table}
                \centering
                \begin{tabular}{|c|c|c|}
                    \hline
                    \textbf{ID} & \textbf{Name} & \textbf{Value} \\ \hline
                    1 & Alpha & 10 \\ \hline
                    2 & Beta & 20 \\ \hline
                    3 & Gamma & 30 \\ \hline
                \end{tabular}
            \end{table}
            \vspace{0.5em}
            % Sample graphic placeholder
            \begin{center}
                \includegraphics[width=0.5\textwidth]{example-image} % Dummy image placeholder
            \end{center}
    \end{columns}
\end{frame}

% --- Frame: Elaborated Table ---
\begin{frame}{Elaborated Table Example}
    \begin{table}
        \centering
        \caption{Dummy Data Table}
        \begin{tabular}{|l|c|r|}
            \hline
            \textbf{Category} & \textbf{Count} & \textbf{Percentage} \\ \hline
            Category A & 15 & 37.50\% \\ \hline
            Category B & 20 & 50.00\% \\ \hline
            Category C & 5  & 12.50\% \\ \hline
        \end{tabular}
    \end{table}
\end{frame}

\section{Section Two}

% --- Frame: Python Code Example ---
\begin{frame}[fragile]{Python Code Examples}
    \framesubtitle{Sample Implementation}
    \begin{centeredcode}
        \begin{minted}[fontsize=\footnotesize,xleftmargin=5em]{python}
            # This is a sample Python function
        def calculate_factorial(n):
            """Calculate the factorial of a number"""
            if n == 0 or n == 1:
                return 1
            else:
                return n * calculate_factorial(n - 1)

        # Example usage
        number = 5
        result = calculate_factorial(number)
        print(f"Factorial of {number} is {result}")
    \end{minted}
    \end{centeredcode}
\end{frame}

% --- Frame: Inline Code Examples ---
\begin{frame}[fragile]{Inline Code Examples}
    \framesubtitle{Code Listing Example}
    \begin{itemize}
        \item Here's a sorting algorithm implementation:
        \begin{minted}[fontsize=\footnotesize, linenos=false]{python}
def quick_sort(arr):
    if len(arr) <= 1: return arr
    pivot = arr[0]
    left = [x for x in arr[1:] if x < pivot]
    right = [x for x in arr[1:] if x >= pivot]
    return quick_sort(left) + [pivot] + quick_sort(right)
        \end{minted}
    \end{itemize}
\end{frame}

\begin{frame}{Inline Code in Paragraphs}
    \framesubtitle{Text with Code Examples}
    
    When working with Python, you can create a list using square brackets like 
    \mintinline{python}{my_list = [1, 2, 3]} or define a dictionary with 
    \mintinline{python}{my_dict = {"key": "value"}}.
    
    \medskip
    
    Function definitions are straightforward: \mintinline{python}{def greet(name): return f"Hello {name}"}
    can be used to create simple greeting functions.
    
    \medskip
    
    For loops are common in Python: \mintinline{python}{for i in range(5): print(i)} will print
    numbers from 0 to 4.
\end{frame}

\begin{frame}{Description List Example}
    \framesubtitle{Terms and Definitions}
    \begin{description}
        \item[CPU] Central Processing Unit - The primary processor that executes instructions
        \item[RAM] Random Access Memory - Temporary storage for running programs
        \item[GPU] Graphics Processing Unit - Specialized processor for rendering graphics
        \item[SSD] Solid State Drive - Fast storage device with no moving parts
    \end{description}
\end{frame}

\backmatter % generate the final slide
\end{document}
